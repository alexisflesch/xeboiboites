\documentclass[11pt,oneside]{article}
\usepackage[utf8]{inputenc}
\usepackage[a4paper,left=2.5cm,right=2.5cm, bottom=2.5cm,top=2.5cm]{geometry}
\usepackage{xeboiboites}
\usepackage{verbatim}
\usepackage{amsmath,amssymb}
\usepackage{lipsum}
\DeclareTextFontCommand{\emph}{\bfseries}


\newboxedtheorem[
small box style={fill=gray!20,draw=black, rounded corners},
big box style={fill=gray!10,draw=orange,thick,rounded corners},
headfont=\bfseries,
thcounter=section]{propbof}{Proposition}{compteurPROP}

\newbreakabletheorem[small box style={draw=orange,fill=blue!20},
    big box style={fill=blue!10,draw=orange}]
    {propc}{Proposition}{somecounter}
    
\newboxedtheorem[small box style={fill=blue!20,draw=black, 
    rounded corners},
    big box style={fill=blue!10,draw=orange,thick,rounded corners},
    headfont=\bfseries]%
    {proposition}{Proposition}{somecounter}    
    
\newboxedtheorem[small box style={fill=blue!20,draw=black, line width=.7pt,
                                    decoration={penciline},decorate},%
    big box style={fill=blue!10,draw=black,thick, 
        decoration={penciline},decorate},
    headfont=\bfseries]%
    {propb}{Proposition}{}


\newboxedequation[big box style={fill=blue!10,%
    thick,decoration=penciline,decorate}]%
    {formula}      
    


\newbreakabletheorem[small box style={draw=orange,fill=blue!20},
    big box style={fill=blue!10,draw=orange},
    broken edges={decoration=zigzag}]
    {propd}{Proposition}{test}    

\newbreakabletheorem[small box style={draw=orange!30!black!20,%
    fill=orange!10!black!2,decoration=penciline, decorate, thick},
    big box style={color=orange!30!black!20,fill=orange!30!black!10,thick},
    broken edges={draw=orange!30!black!20,thick,fill=orange!20!black!5, 
        decoration={random steps, segment length=.5cm,%
        amplitude=1.3mm},decorate},%
    other edges={decoration=penciline,decorate,thick}]%
    {parchment}{Parchment}{test}    

\newparchment[small box style={draw=orange!30!black!20,%
    fill=orange!10!black!2,decoration=penciline, decorate, thick},
    big box style={color=orange!30!black!20,fill=orange!30!black!10,thick},
    broken edges={draw=orange!30!black!20,thick,fill=orange!20!black!5, 
        decoration={random steps, segment length=.4cm,%
        amplitude=1.7mm},decorate},%
    other edges={decoration=penciline,decorate,thick}]%
    {parchmentb}{Parchment}{}     

\newspanning[image=dessins/bulb,headfont=\bfseries,%
    spanning style={very thick,decoration=penciline,decorate}]%
    {method}{Method}{}
  
\newspanning[image=dessins/poisson,headfont=\itshape,%
    spanning style={very thick,decoration=penciline,decorate}]%
    {test}{Test}{}
    
\parindent 0pt
\title{\Huge\bfseries xeboiboites}
\author{Alexis Flesch}
\begin{document}
\maketitle

\tableofcontents\clearpage

\section{Introduction}

This document describes the usage of the package \emph{xeboiboites}, which is a 
rewrite of \emph{boiboites}. It allows breakable boxes as well as a few new 
environments. This work is mainly based on:
\begin{verbatim}
http://www.texample.net/tikz/examples/boxes-with-text-and-math/
http://tex.stackexchange.com/questions/39296/simulating-hand-drawn-lines
http://www.texample.net/tikz/examples/framed-tikz/
\end{verbatim}
It is released under the CC BY-SA licence.

\section{Non-breakable boxed theorems}

\subsection{Overview}

Creating a new boxed environment hasn't changed much since \emph{boiboites}, 
except you now have control over (almost) everything tikz-related. Issuing:

\begin{verbatim}
\newboxedtheorem[options]{proposition}{Proposition}{somecounter}
\end{verbatim}

where \verb!proposition! is the name of the new environment, \verb!Proposition! 
is the way it will be typeset and \verb!somecounter! is a counter name. If the 
counter doesn't exist, it will be created, otherwise the new environment will 
use the already existing counter. If no counter is given (that is, if you leave 
the braces empty), then the proposition will not be numbered. As for the 
options:
\begin{itemize}
    \item \verb!thcounter=chapter! will reinitialize the counter at each new 
        chapter (you can of course use section, etc...).
    \item \verb!size! is the width of the box, default is 
        \verb!size=.9\textwidth!.
    \item \verb!headfont! is issued before writing the title of the box. You 
        might want to set for example \verb!headfont=\bfseries\large!.
    \item \verb!small box style! is the style of, well, the small box in which 
        the title will appear (see below for examples).
    \item \verb!big box style! is the style of the main box in which the 
        proposition will appear (see below for examples).
\end{itemize}


\subsection{Examples}


\subsubsection{A simple box with rounded corners}
\begin{verbatim}
\newboxedtheorem[small box style={fill=blue!20,draw=black, rounded corners},
    big box style={fill=blue!10,draw=orange,thick,rounded corners},
    headfont=\bfseries]%
    {proposition}{Proposition}{somecounter}
\end{verbatim}


\begin{proposition}[Law of large numbers]
     Let $(X_n)_{n\in \mathbb{N}}$ be a sequence of mutually independent random 
    variables such that $X_1 \in L^1$. Then :
     \[\frac{1}{n} \sum_{i=1}^n X_i \overset{\textnormal{p.s.}}{\longrightarrow}
         \mathbb{E} (X_1) .\]
\end{proposition}


\subsubsection{An example with a hand drawn effet}

\begin{verbatim}
\newboxedtheorem[small box style={fill=blue!20, draw=black, line width=.7pt,
    decoration={penciline},decorate},
    big box style={fill=blue!10,draw=black,thick,
    decoration={penciline},decorate},
    headfont=\bfseries]%
    {proposition}{Proposition}{}
\end{verbatim}


\begin{propb}[Law of large numbers]
     Let $(X_n)_{n\in \mathbb{N}}$ be a sequence of mutually independent random 
     variables such that $X_1 \in L^1$. Then :
     \[\frac{1}{n} \sum_{i=1}^n X_i \overset{\textnormal{p.s.}}{\longrightarrow}
         \mathbb{E} (X_1) .\]
\end{propb}

\subsubsection{An example with no title}

If for some reason you want to frame the results (say, a formula) using the same 
style as the one for your theorems, here is a way to do it:
\begin{verbatim}
\newboxedequation[big box style={fill=blue!10,%
    thick,decoration=penciline,decorate}]%
    {formula}
\end{verbatim}

\begin{formula}
    \[ \forall x\in\mathbb{R},\quad\sin(x) = x-\frac{x^3}{3!} + o(x^4).\]
\end{formula}


\section{Breakable boxed theorems}

In some situations, you might want to have a box that spans over multiple pages.
This is possible using the new macro:
\begin{verbatim}
\newbreakabletheorem[options]{proposition}{Proposition}{somecounter}
\end{verbatim}
The \verb!options! are exactly the same except as before except you also
have control over the style of the edged of the big box (see examples below).



\subsection{Examples}

\begin{propc}
    \lipsum[4-7]
\end{propc}

The above example was created using:
\begin{verbatim}
\newbreakabletheorem[small box style={draw=orange,fill=blue!20},
    big box style={fill=blue!10,draw=orange}]
    {proposition}{Proposition}{}
\end{verbatim}

You have the control over the decoration of the edges of the box, so you can,
for example, add a decoration to the bottom/top edges when a pagebreak occurs:
\begin{verbatim}
\newbreakabletheorem[small box style={draw=orange,fill=blue!20},
    big box style={fill=blue!10,draw=orange},
    broken edges={decoration=zigzag}]
    {proposition}{Proposition}{}  
\end{verbatim}

\begin{propd}
    \lipsum[4-7]
\end{propd}


To apply a style to the edges after/before a pagebreak, use the
parameter \verb!broken edges!. Use \verb!other edges! to customize the
"normal" edges. Here is a more sophisticated example:

\begin{verbatim}
\newbreakabletheorem[small box style={draw=orange!30!black!20,%
    fill=orange!10!black!2,decoration=penciline, decorate, thick},
    big box style={color=orange!30!black!20,fill=orange!30!black!10,thick},
    broken edges={draw=orange!30!black!20,thick,fill=orange!20!black!5, 
        decoration={random steps, segment length=.5cm,%
        amplitude=1.3mm},decorate},%
    other edges={decoration=penciline,decorate,thick}]%
    {parchment}{Parchment}{}
\end{verbatim}


\begin{parchment}
    \lipsum[10-21]
\end{parchment}

Note that this might result in an overfull $\backslash$vbox due to the random
effect and the fact that the framed environment has been
resized with \verb!\hsize\size\FrameRestore!. I don't
know how to get rid of it, but the parchment command below is a possible workaround
(although compilation time is longer).



\section{Breakable parchments}

This is a rewrite of:
\begin{verbatim}
http://www.texample.net/tikz/examples/framed-tikz/
\end{verbatim}

as is almost everything here that concerns "breakable" environments. You can create
a new parchment using the command:

\begin{verbatim}
\newparchment[options]{proposition}{Proposition}{somecounter}
\end{verbatim}

The options are the same as the ones for breakable theorems. The difference
is that another layer has been added to make it look like the box was torn whenever
a pagebreak occurs. I am not responsible for this great design, I only tweaked
the code a little.

\begin{parchmentb}
    \lipsum[10-21]
\end{parchmentb}

The corresponding code:
\begin{verbatim}
\newparchment[small box style={draw=orange!30!black!20,%
    fill=orange!10!black!2,decoration=penciline, decorate, thick},
    big box style={color=orange!30!black!20,fill=orange!30!black!10,thick},
    broken edges={draw=orange!30!black!20,thick,fill=orange!20!black!5, 
        decoration={random steps, segment length=.4cm,%
        amplitude=1.7mm},decorate},%
    other edges={decoration=penciline,decorate,thick}]%
    {parchment}{Parchment}{}      
\end{verbatim}

Once again, \verb!\hsize\size\FrameRestore! is making things difficult
with the random effect and changing the random parameters might result
in strange things (but no badbox as the \verb!overlay! option of tikz
was used).

If you use the option \verb!opacity=0! on the \verb!broken edges!, you'll
get a similar result as the one presented in the previous section (but
with a longer compilation time, of course).


\section{A framed-like environment with a rule in the margin}

Last but not least, an environment that can span over multiple pages with an
image in the margin and a vertical rule all along the way. The bulb image in 
this example was borrowed to \texttt{openclipart.org}.

To create a new environment "method" for example, use:
\begin{verbatim}
\newspanning[options]{method}{Method}{somecounter}
\end{verbatim}
 The options here are
\begin{itemize}
    \item \verb!image! is the path to an image
    \item \verb!headfont! is issued before writing the title of the
        environment. You 
        might want to set for example \verb!headfont=\bfseries\large!.
    \item \verb!spanning style! is the style of the vertical rule
\end{itemize}

\begin{method}
    \lipsum[1-10]
\end{method}

The corresponding code:
\begin{verbatim}
\newspanning[image=dessins/bulb,headfont=\bfseries,%
    spanning style={very thick,decoration=penciline,decorate}]%
    {method}{Method}{}
\end{verbatim}
where \verb!dessins/bulb! is the path to the bulb image on my computer.


\end{document}
